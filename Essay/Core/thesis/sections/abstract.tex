\documentclass[../main.tex]{subfiles}
\begin{document}
\begin{abstract}
  Australia is undergoing huge wildfires in every state. To protect people and safety and property, we establish a model to use two types of drones to help Country Fire Authority (CFA) conduct “Rapid Bushfire Response”, 
  and front-line personnel communicate with Emergency Operations Center (EOC).
  \\ We use Victoria Fire Report Data, using fire report places in certain 
  time period to represent locations where fires happened on average.
   This allows us to take into account of factors such as fire size and 
   frequency, economic cost and safety, weighted region area covered by drones.  \\ 
   We set Fast Response Model for deployment of drones for fast response, we quantify the
   coverage with a weighted quantity. We build a model to investigate its' relationship with 
   the economic cost. For a certain coverage by drone, we devise a strategy to distribute SSAs by using
    k-means with special distance function and we use minimum spanning tree to 
   distribute repeaters to connect them. We show the mathematical properties to such distance to ensure the correctness of our algorithm.
   \\ We set Fire Prediction Model for second part of problem. we divide Victoria into several zones, 
   and use statistics of zones in different time to form a time series. 
   We predict the time series using convolutional Long Short-Term Memory (convLSTM). \\ 
   We set Pearl Model and Spur Model for Deployment of drones for front-line personnel in different circumstances, 
   we use separate deployment strategies for small and big sized fire considering the effect of terrain. \\ 
   The sensitivity analysis shows robustness in our model. Meanwhile, we combine all the models to finish the annotated
  Budget Request to help CFA with acceptable cost.
  \begin{keywords}
    Clustering, Minimum Spanning Tree, ConvLSTM, Terrian;
  \end{keywords}
\end{abstract}
\maketitle
\end{document}
